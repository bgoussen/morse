\documentclass{article}

\usepackage{fullpage}
\usepackage{amsmath}
\usepackage{amsfonts}

%\VignetteIndexEntry{Models}
\title{Models in MORSE package}

\usepackage{Sweave}
\begin{document}
\Sconcordance{concordance:modelling.tex:/home/virgile/Documents/morse/vignettes/modelling.Snw:%
1 9 1 1 0 413 1}


\maketitle

This document describes the statistical models used in MORSE to
analyze survival and reproduction data, and as such serves as a
mathematical specification of the package. For a more practical
introduction, please consult the ``Tutorial'' vignette; for
information on the structure and contents of the library, please
consult the reference manual.

Model parameters are estimated using Bayesian inference,
where posterior distributions are computed from the
likelihood of observed data and prior distributions on the
parameters. These priors are specified after each model description.

\section{Survival bioassays}
In a survival bioassay, subjects are exposed to a controlled dose of a
contaminant over a given period of time and the number of surviving
individuals is measured at certain time points during exposition. In
most standard bioassays, the dose is held constant throughout the
whole experiment, which we will assume here. An experiment is
generally replicated several times and also repeated for various
levels of the contaminant.

In so-called \emph{final time} bioassays, the mortality is measured
only at the end of the experiment. The chosen time point is called
\emph{target time}. Let us see how this particular case is handled in
MORSE.

\subsection{Analysis of final time survival bioassays}

A dataset from a final time survival bioassay is a collection $D = \{
(c_i, n_i^{init}, n_i) \}_i$ of experiments, where $c_i$ is the tested
concentration, $n_i^{init}$ the initial number of individuals and
$n_i$ the number of individuals at the target time. Triplets such that
$c_i = 0$ correspond to control experiments.

\paragraph{Modelling} In the particular case of endpoint assays, the
model used in MORSE is defined as follows. Let $t$ be the target time
in days. We suppose the \emph{mean survival rate after $t$ days} is
given by a function $f$ of the contaminant level $c$. We also suppose
that the death of two individuals are two independent events. Hence,
given an initial number $n^{init}_i$ of individuals in the bioassay,
we obtain that the number $N_i$ of surviving individuals at time $t$
follows a binomial distribution:
$$
N_i \sim \mathcal{B}(n^{init}_i, f(c_i))
$$
Note that this model neglects inter-replicate variations, as a given
concentration of pollutant implies a fixed value of the survival
rate. There may be various possibilities for $f$. In MORSE we assume:
$$
f(c) = \frac{d}{1 + (\frac{c}{e})^b}
$$
where $b$, $e$ and $d$ are (positive) parameters. In particular $d$
corresponds to the survival rate in absence of pollutant and $e$
corresponds to the $LC_{50}$. The parameter $b$ is related to the
effect intensity of the contaminant.

\paragraph{Inference} Posterior distributions for parameters $b$,
$d$ and $e$ are estimated using JAGS with the following priors:
\begin{itemize}
\item we assume the range of tested concentrations in an
  experiment is chosen to contain the $LC_{50}$ with high
  probability. More formally, we choose:
  $$\log_{10} e \sim \mathcal{N}(\frac{\log_{10} (\min_i c_i) + \log_{10} (\max_i c_i)}{2}, \frac{\log_{10} (\max_i c_i) - \log_{10} (\min_i c_i)}{4})$$
  which implies $e$ has a probability slightly higher than 0.95 to lie
  between the minimum and the maximum tested concentration.
\item we choose a quasi non-informative prior distribution for the
  shape parameter $b$:
  $$\log_{10} b \sim \mathcal{U}(-2,2)$$
\end{itemize}

\begin{align*}
\end{align*}

The prior on $d$ is chosen as follows: if we observe
no mortality in control experiments then we set $d = 1$, otherwise we
assume a uniform prior for $d$ between 0 and 1.

\subsection{Toxico-kinetic toxico-dynamic modeling}

For datasets featuring time series measurements, more complete models
can be used to estimate the effect of a contaminant on survival. We
assume the bioassay consists in exposing an initial number $n_i^0$ of
individuals to a constant dose $c_i$ of contaminant, and following the
number $n_i^k$ of survivors at time $t_k$ (with $t_0 < t_1 < \dots <
t_m$ and $t_0 = 0$), providing a collection $D = {(c_i, t_k,
  n_i^k)}_{i,k}$ of experiments. In MORSE, we implement a
toxico-kinetic toxico-dynamic (TKTD) model variant known as the
\emph{reduced stochastic death} model~\cite{nyman2012}, which we
describe now.

\newcommand*{\diffdchar}{\mathrm{d}}
\newcommand*{\dd}{\mathop{\diffdchar\!}}
\newcommand*{\NEC}{\mbox{\it NEC}}
\newcommand*{\Cint}{C^{\mbox{\tiny INT}}}
\newcommand*{\tNEC}{t^{\tiny \NEC}}


\paragraph{Modelling} The number of survivors at time $t_k$ given the
number of survivors at time $t_{k-1}$ is assumed to follow a binomial
distribution:
$$
N_i^k \sim \mathcal{B}(n_i^{k-1}, f_i(t_{k-1}, t_k))
$$
where $f$ is the conditional probability of survival at time $t_k$
given survival at time $t_{k-1}$. Denoting $S_i(t)$ the probability
of survival at time $t$, we have:
$$
f_i(t_{k-1}, t_k) = \frac{S_i(t_k)}{S_i(t_{k-1})}
$$
$S_i$ can be calculated by integrating the \emph{instantaneous mortality
rate} $h_i$:
\begin{equation}
  \label{eq:survivaldef}
  S_i(t) = \exp \left(\int_0^t - h_i(u)\mbox{d}u \right)
\end{equation}
In the model, the function $h_i$ is expressed using the internal
concentration of contaminant (that is, the concentration inside an
individual) $\Cint_i(t)$, more precisely:
$$
h_i(t) = k_s \max(\Cint_i(t) - \NEC, 0) + m_0
$$
where:
\begin{itemize}
\item $k_s$ is called \emph{killing rate} and expressed in
  concentration$^{-1}$.time$^{-1}$,
\item $\NEC$ is the so-called \emph{no effect concentration} and
  represents a concentration threshold under which the contaminant has
  no effect on individuals,
\item $m_0$ is the \emph{background mortality} (mortality in absence of
  contaminant), expressed in time$^{-1}$.
\end{itemize}

\noindent The internal concentration is assumed to be driven by the external
concentration, following:
$$
\frac{\dd \Cint_i}{\dd t}(t) = k_d (c_i - \Cint_i(t))
$$
which we can integrate to obtain:
\begin{equation}
\label{eq:cint}
\Cint_i(t) = c_i(1 - e^{-k_d t})
\end{equation}
assuming $\Cint_i(0) = 0$. We call the parameter $k_d$ of
Eq.~\ref{eq:cint} the ``dominant rate'' (expressed in time$^{-1}$). It
represents the speed at which the internal concentration in
contaminant converges to the external concentration. The model could
be equivalently written using an internal damage instead of an
internal concentration as a dose metric~\cite{jager2011}. With the
data we assume is available, it is not possible to distinguish
dynamical effects due to toxico-kinetic elimination and damage
recovery.

\bigskip

\noindent In the case $c_i <
\NEC$, the individuals are never affected by the contaminant:
\begin{equation}
  \label{eq:bgsurvival}
  S_i(t) = exp( - m_0 t )
\end{equation}
When $c_i > \NEC$, it takes time $\tNEC_i$ before the internal
concentration reaches $\NEC$, where:
$$
\tNEC_i = - \frac{1}{k_d} \log (1 - \frac{\NEC}{c_i}).
$$
Before that happens, Eq.~\ref{eq:bgsurvival} applies, while for $t >
\tNEC_i$, integrating Eq.~\ref{eq:survivaldef} results in:
$$
S_i(t) = \exp(- m_0 t - k_s(c_i - \NEC)(t - \tNEC_i) - \frac{k_s c_i}{k_d}(e^{- k_d t} - e^{-k_d \tNEC_i}))
$$

In brief, given values for our four parameters $m_0$, $k_s$, $k_d$ and
$\NEC$, we can simulate trajectories by using $S_i$ to compute
conditional survival probabilities. In MORSE those parameters are
estimated using Bayesian inference. The next paragraph describes our
choice of priors.

\paragraph{Inference} Posterior distributions for $m_0$, $k_s$, $k_d$ and
$\NEC$ are computed with JAGS. We set prior distributions on those
parameters based on the actual experimental design used in a
bioassay. For instance, we assume the NEC is close to the range of
tested concentrations. For each parameter $\theta$, we derive in a similar
manner a minimum ($\theta^{\min}$) and a maximum ($\theta^{\max}$)
value and state that the prior on $\theta$ is a log-normal
distribution. More precisely, we choose:
$$
\log_{10} \theta \sim \mathcal{N}(\frac{\log_{10} \theta^{\min} + \log_{10} \theta^{\max}}{2},
                                 \frac{\log_{10} \theta^{\max} - \log_{10} \theta^{\min}}{4})
$$
With this choice, $\theta^{\min}$ and $\theta^{\max}$ correspond to the
2.5 and 97.5 percentiles of the prior distribution on $\theta$. For
each of our parameters, this gives:
\begin{itemize}
\item $\NEC^{\min} = \min_{i, c_i \neq 0} c_i$  and $\NEC^{\max} = \max_i c_i$,
  which amounts to say that the NEC is most probably contained in the
  range of experimentally tested concentrations
\item for the mortality rate $m_0$, we assume a maximum value
  corresponding to situations where half the indivuals are lost at the
  first observation time in the control (time $t_1$), that is:
  $$
  e^{- m_0^{\max} t_1} = 0.5 \Leftrightarrow m_0^{\max} = - \frac{1}{t_1} \log 0.5
  $$
  To derive a minimum value for $m_0$, we set the maximal survival
  probability at the end of the bioassay in control condition to
  0.999, which corresponds to saying that the average lifetime of the
  considered species is at most a thousand times longer than the
  duration of the experiment. This gives:
  $$
  e^{- m_0^{\min} t_m} = 0.999 \Leftrightarrow m_0^{\min} = - \frac{1}{t_m} \log 0.999
  $$
\item $k_d$ is the parameter describing the speed at which the
  internal concentration of contaminant equilibrates with the external
  concentration. We suppose its value is such that the internal
  concentration can at most reach 99.9\% of the external concentration
  within the first observation point, implying the maximum value for
  $k_d$ is:
  $$
  1 - e^{- k_d^{\max} t_1} = 0.999 \Leftrightarrow k_d^{\max} = - \frac{1}{t_1} \log 0.001
  $$
  For the minimum value, we assume the internal concentration should
  at least have risen to 0.1\% of the external concentration at the
  end of the experiment, which gives:
  $$
  1 - e^{- k_d^{\min} t_m} = 0.001 \Leftrightarrow k_d^{\min} = - \frac{1}{t_m} \log 0.999
  $$
\item $k_s$ is the parameter relating the internal concentration of
  contaminant to the instantaneous mortality. To fix a maximum value,
  we state that between the closest two tested concentrations, the
  survival probability at the first observation point should not be
  divided by more than one thousand, assuming (infinitely) fast
  equilibration of internal and external concentrations. This last
  assumption means we take the limit $k_d \rightarrow + \infty$ and
  approximate $S_i(t)$ with $\exp(- (m_0 + k_s(c_i -
  \NEC))t)$. Denoting $\Delta^{\min}$ the minimum difference between
  two tested concentrations, we obtain:
  $$
  e^{- k_s^{\max} \Delta^{\min} t_1} = 0.001 \Leftrightarrow k_s^{\max} = - \frac{1}{\Delta^{\min} t_1} \log 0.001
  $$
  Analogously we set a minimum value for $k_s$ saying that the
  survival probability at the last observation time for the maximum
  concentration should not be higher than 99.9\% of what it is for the
  minimal tested concentration. For this we assume again $k_d
  \rightarrow + \infty$. Denoting $\Delta^{\max}$ the maximum
  difference between two tested concentrations, this leads to:
  $$
  e^{- k_s^{\min} \Delta^{\max} t_m} = 0.001 \Leftrightarrow k_s^{\min} = - \frac{1}{\Delta^{\max} t_m} \log 0.999
  $$
\end{itemize}

\section{Reproduction bioassays}

In a reproduction bioassay, we observe the number of offspring
produced by a population of adult individuals subjected to a certain
dose of a contaminant over a given period of time. The offspring
(young individuals, clutches or eggs) are regularly counted and
removed from the medium at each observation, so that the reproducing
population cannot increase. It can decrease however, if some
individuals die during the experiment. The same procedure is usually
repeated with various concentrations of contaminant, in order to
establish a quantitative relationship between the reproduction rate
and the concentration of contaminant in the medium.

As mentionned already, it is often the case that part of the
individuals of an bioassay die during the observation period. In
previous approaches, it was proposed to consider the cumulated number
of reproduction outputs without accounting for
mortality~\cite{OECD2004,OECD2008}, or to exclude replicates where
mortality occurred~\cite{OECD2012}.  However, individuals may have
reproduced before dying and thus contributed to the observed
response. In addition, individuals dying the first are the most
sensitive, so the information on reproduction of these prematurely
dead individuals is valuable and ignoring it is likely to bias the
results in a non-conservative way. This is particularly critical at
high concentrations, when mortality may be very high.

In a bioassay, mortality is usually regularly recorded, \textit{i.e}.
at each timepoint when reproduction outputs are counted.  Using these
data, we can approximately estimate for each individual the period it
has stayed alive (which we assume coincides with the period it may
reproduce).  As commonly done in epidemiology for incidence rate
calculations, we can then calculate, for one replicate, the total sum
of the periods of observation of each individual before its death (see
next paragraph). This sum can be expressed as a number of
individual-days. Hence, reproduction can be evaluated through the
number of outputs per individual-day.

In the following, we denote $M_{ijk}$ the observed number of surviving
individuals for concentration $c_i$, replicate $j$ and time $t_k$.

\subsection{Estimation of the effective observation time}

We define the effective observation time as the sum for all
individuals of the time they spent alive in the experiment. It is
counted in individual-days and will be denoted $NID_{ij}$ for
concentration $c_i$ and replicate $j$. As mentionned earlier,
mortality is observed at particular time points only, so the real life
time of an individual is unknown and in practice we use the following
simple estimation: if an individual is alive at $t_k$
but dead at $t_{k+1}$, its real life time is approximated as
$\frac{t_{k+1}+t_k}{2}$.

With this assumption, the effective observation time for concentration
$c_i$ and replicate $j$ is then given by:
$$
NID_{ij} =   \sum_k M_{ij(k+1)} (t_{k+1} - t_k)
           + (M_{ijk} - M_{ij(k+1)})(\frac{t_{k+1}+t_k}{2} - t_k)
$$

\subsection{Target time analysis}
In this paragraph, we describe our so-called ``target time analysis'',
where we model the cumulated number of offspring up to a target time
as a function of pollutant concentration and effective observation
time in this period (cumulated life times of all individuals in the
experiment, as described above). A more detailed presentation can be
found in \cite{delignette2014}.

We keep the convention that the index $i$ is used for concentration
levels and $j$ for replicates. The data will therefore correspond to a
set $\{(nid_{ij}, n_{ij})\}_i$ of pairs, where $nid_{ij}$ denotes the
effective observation time and $n_{ij}$ the number of reproduction
output. These observations are supposed to be drawn independently from
a distribution that is a function of the level of contaminant $c_i$.


\paragraph{Modelling} We assume here that the effect of the considered
contaminant on the reproduction rate\footnote{that is, the number of
  reproduction outputs during the experiment per individual-day} does
not depend on the exposure time, but only on the concentration of the
contaminant. More precisely, the reproduction rate in an
experiment with a concentration $c_i$ of contaminant is modelled by a
three-parameter log-logistic model, that writes as follows:
\[
f(c;\theta)=\frac{d}{1+(\frac{c}{e})^b} \quad \textrm{with} \quad
\theta=(e,b,d)
\]
Here $d$ corresponds to the reproduction rate in absence of
contaminant (control condition), and $e$ to the value of the
$EC_{50}$, that is the concentration dividing the average number of
offspring by two with respect to the control condition.  Now the
number of reproduction outputs $N_{ij}$ for concentration $c_i$ in
replicate $j$ can be modelled using a Poisson distribution:
$$
N_{ij} \sim Poisson(f(c_i ; \theta) \times NID_{ij})
$$
This model is later referred to as ``Poisson model''.  If there
happens to be a non-negligible variability of the reproduction rate
between replicates for a some fixed concentration, we propose a second
model, named ``gamma-Poisson model'', stating that:
$$
N_{ij} \sim Poisson(F_{ij} \times NID_{ij})
$$
where the reproduction rate $F_{ij}$ for at $c_i$ in replicate $j$ is
a random variable following a gamma distribution. Introducing a
dispersion parameter $\omega$, we assume that:
$$
F_{ij} \sim gamma(\frac{f(c_i;\theta)}{\omega}, \frac{1}{\omega})
$$
Note that a gamma distribution of parameters $\alpha$ and $\beta$ has
mean $\frac{\alpha}{\beta}$ and variance $\frac{\alpha}{\beta^2}$,
that is here $f(c_i;\theta)$ and $\omega f(c_i;\theta)$
respectively. Hence $\omega$ can be considered as an overdispersion
parameter (the greater its value, the greater the inter-replicate
variability)


\paragraph{Inference} Posterior distributions for parameters $b$,
$d$ and $e$ are estimated using JAGS with the following priors:
\begin{itemize}
\item we assume the range of tested concentrations in an
  experiment is chosen to contain the $EC_{50}$ with high
  probability. More formally, we choose:
  $$\log_{10} e \sim \mathcal{N}(\frac{\log_{10} (\min_i c_i) + \log_{10} (\max_i c_i)}{2}, \frac{\log_{10} (\max_i c_i) - \log_{10} (\min_i c_i)}{4})$$
  which implies $e$ has a probability slightly higher than 0.95 to lie
  between the minimum and the maximum tested concentration.
\item we choose a quasi non-informative prior distribution for the
  shape parameter $b$:
  $$\log_{10} b \sim \mathcal{U}(-2,2)$$

\item
  as $d$ corresponds to the reproduction rate without contaminant, we
  set a normal prior $\mathcal{N}(\mu_d,\sigma_d)$ using the data:
  \begin{align*}
    \mu_d & = \frac{1}{r_0} \sum_j \frac{n_{0j}}{nid_{0j}}\\
    \sigma_d & = \sqrt{\frac{\sum_j (\frac{n_{0j}}{nid_{0j}} - \mu_d)^2}{r_0(r_0 - 1)}}\\
  \end{align*}
  where $r_0$ is the number of replicates in the control
  condition. Note that since they are used to estimate the prior
  distribution, the data from the control condition are not used in the
  fitting phase.

\item we choose a quasi non-informative prior distribution for the
  $\omega$ parameter of the gamma-Poisson model:
$$log_{10}(\omega) \sim \mathcal{U}(-4,4)$$
\end{itemize}
\begin{align*}
\end{align*}

For a given dataset, the procedure implemented in MORSE will fit both
models (Poisson and gamma-Poisson), and use an information criterion
known as Deviance Information Criterion (DIC) to choose the most
appropriate. In situations where overdispersion (that is
inter-replicate variability) is negligible, using the Poisson model
will provide more reliable estimates. That is why a Poisson model is
preferred unless the gamma-Poisson model has a sufficiently lower DIC
(in practice we require a difference of 10).

\bibliographystyle{plain}
\bibliography{biblio}
\end{document}
